\section{总结}
\label{sec:conclusion}

传统数字图像处理方法使用成熟的技术处理问题,如特征描述子(SIFT、SUR、BRIEF 等)。在深度学习兴起前,图像分类等任务需要用到特征提取步骤,特征即图像中``有趣''、描述性或信息性的小图像块。这一步可能涉及多种数字图像处理算法,如边缘检测、角点检测或阈值分割算法。从图像中提取出足够多的特征后,这些特征可形成每个目标类别的定义(即``词袋'')。部署阶段中,在其他图像中搜索这些定义。如果在一张图像中找到了另一张图像词袋中的绝大多数特征,则该图像也包含同样的目标(如椅子、马等)。传统方法的缺陷是从每张图像中选择重要特征是必要步骤。而随着类别数量的增加,特征提取变得越来越麻烦。要确定哪些特征最能描述不同的目标类别,取决于研究人员的判断和长期试错。此外,每个特征定义还需要处理大量参数,所有参数必须由人来进行调整。

而深度学习的快速发展和设备能力的改善(如算力、内存容量、能耗、图像传感器分辨率和光学器件)提升了视觉应用的性能和成本效益,并进一步加快了此类应用的扩展。与传统方法相比,深度学习可以帮助 研究人员在图像分类、语义分割和目标检测等任务上获得更高的准确率。由于深度学习所用的神经网络是训练得到而非编程得到,因此使用该方法的应用所需的专家分析和微调较少,且能够处理目前系统中的海量可用视频数据。深度学习引入了端到端学习的概念,即向计算机提供的图像数据集中的每张图像均已标注目标类别。因而深度学习模型基于给定数据``训练''得到,其中神经网络发现图像类别中的底层模式,并自动提取出对于目标类别最具描述性和最显著的特征。

但这并不意味着传统方法的没落,在本文的调研和分析中,可以看到即使在深度学习大放异彩的时代,仍在一定程度上依赖于传统方法的使用,甚至从中获得理论指导或思路启迪,如正交变换、高斯滤波等。传统数字图像处理方法和深度学习方法之间存在一定的权衡。传统方法成熟、透明,且为性能和能效进行过优化;深度学习提供更好的准确率和通用性,但消耗的计算资源也更大。混合方法结合传统方法和深度学习,兼具这两种方法的优点,更加适用于需要快速实现的高性能系统。