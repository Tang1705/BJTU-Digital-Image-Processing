\section{图像增强} 
\label{sec:image_enhancement}

图像增强是指按特定的需要突出一幅图像中的某些信息,同时削弱或去除某些不需 要信息的处理方法。其主要目的是使处理 后的图像对某种特定的应用来说,比原始 图像更适用。因此,这类处理是为了某种应用而去改善图像质量。处理的结果使图像更适合于人的视觉特性或机器识别系统。增强处理不能增强(加)原始图像的信息,其 结果只是增强对某种信息的辨别能力,而这种处理有可能损失一些其他信息。图像增强是数字图像处理的基本内容之一。本文将以低光照图像增强为研究对象,对其传统方法和基于深度学习的方法进行调研。

图像作为感知世界的重要媒介,其质量往往决定了算法的性能。然而在图像捕获过程中,往往存在多种不可控的物理因素,导致图像质量受损,进而影响了高层视觉任务应用过程中信息的获取与利用,如人脸识别、语义分割等。在多种影响图像质量的因素当中,低光照因素较为常见且难以避免,例如阴天、夜晚等场景。为了在低光照环境下获得肉眼可见且信息丰富的高质量图像,一种直接的方式是通过物理成像过程的参数设置来提高成像质量。但单纯依赖物理成像过程的调整依然难以获得理想的图像。因此构建智能化算法以提升图像质量,已成为低光照图像增强领域的研究热点之一。

\textbf{图 \ref{fig:fig3}} 展示了不同场景下的低光照图像,从人眼的观测角度来看,几乎无法获取到有价值的信息,尤其是对于极具挑战的例子而言。事实上对于计算机而言,在这些肉眼不可见的矩阵数据中,其数值上存在的差异性是反映图像信息的关键。换言之,人眼不可见只是片面的感受,而计算机能够从图像的数值分布层面的差异和联系出发,对图像有更加清晰的认知。总之,如何利用机器学习算法实现肉眼可见的信息转换是现有低光照图像增强技术的基本目的。

\begin{figure}[!ht]
	\centering
	\includegraphics[width=\linewidth]{3}
	\caption{低光照图像示例}
	\label{fig:fig3}
\end{figure}

根据算法设计理念不同,现有的低光照图像增强算法可分为3类,即基于分布映射的方法,基于模型优化的方法和基于深度学习的方法。其中,基于分布映射的方法关注低光照观测的像素分布情况,致力于利用曲线变换、直方图均衡化等手段改善图像的像素分布,从而提高图像亮度与清晰度。此种类型技术与问题本身的成像过程无关,由于缺乏对于低光照图像自身对于光照需求的建模以及忽略了分布内在的联系,以致于无法有效区分语义,进而导致该类技术生成的结果往往存在颜色失真以及细节异常等视觉不友好的现象。基于模型优化的方法隶属于经典的图像处理技术范畴,其核心在于以物理成像规律导出的数据项作为基本成分,以设计刻画目标变量的正则项作为关键,并进一步利用现有的优化技术进行求解获得基于迭代的算法流程。然而,限于设计的先验的刻画能力,这些传统算法往往会产生曝光不足、色彩不饱和以及伪影或噪声明显的问题。

为了克服上述传统方法的弊端,受大数据时代的影响,通过启发式设计网络结构建立低光照输入与增强输出之间的关系,已成为一种主流的低光照图像增强模式。多种端到端的深度学习技术被设计来解决低光照图像增强任务。其中,引入特定任务的物理原理和先验正则项成为网络结构设计的主流思想。最具代表性的数据驱动型深度网络构建了“三阶段”架构。首先通过分解网络粗略估计了初始光照和反射,然后分别采用设计的架构进一步优化了这两个组件,但是,这种方式容易引起过度曝光现象,因此Retinex理论被进一步考虑以解决过曝问题。然而实际上,该过程严格执行基于模型优化的求解步骤,忽略了深度网络自身的强大推理能力。此外一些工作致力于构建低光照输入与清晰图像之间的显式连接,由于缺乏对于物理规律的利用,往往不可避免地产生其他衍生的影响视觉质量的因素。


\noindent\textbf{基于分布映射的方法}~映射低光照输入的分布以放大较小的值(显示为暗)是解决低光照图像增强的一种最直观的思路。直方图均衡化和基于S型曲线的方法是此类方法的两种代表性工作。常规的直方图均衡化方法生成结果往往会存在曝光不当、细节损失和颜色失真等问题。为此,设计了一系列基于直方图均衡化的改进版本以改善上述缺点。例如,Kim \cite{kim1997contrast} 开发了双直方图均衡化方法,Wang 等 \cite{DBLP:journals/tce/WangCZ99} 设计了二元子图像直方图均衡化方法来实现曝光的自然化处理。为了处理细节损失,Pizer 等 \cite{pizer1987adaptive} 构造了自适应直方图均衡方法,Pisano 等 \cite{DBLP:journals/jdi/PisanoZHDJMBP98}设计了对比度自适应的直方图均衡化方法。然而,由于分布映射过程中缺乏对于语义信息的识别与利用,现有的基于分布映射的方法仍然存在颜色失真等影响增强结果观感的现象。

伽玛校正是最著名的基于S型曲线的图像亮度校正技术之一,它具有映射亮度水平以补偿显示设备的非线性亮度的功能。然而对于低光照图像增强而言,伽马校正的增强结果极其不自然且不真实,尤其是在曝光水平和细节表现上。为了克服这些问题,一系列改进版本相继提出。在Bennett和McMillan \cite{DBLP:journals/tog/BennettM05} 的方法中,使用双边滤波分解低光观测,随后采用不同参数设置的S型曲线方法处理分解层,并进行重新组合。Yuan和Sun \cite{DBLP:conf/eccv/Yuan012}提出试图对通过分割输入而生成的每个子区域执行S型曲线功能。总体而言,现有的基于S型曲线的方法,曝光不均匀现象仍然是存在的最大的问题。

\noindent\textbf{基于模型优化的方法}~Retinex理论 \cite{land1971lightness} 为增强弱光图像的过程提供了直观的物理描述。该理论假设可以通过去除低光输入的光照来获得期望的正常图像(即反射图)。Retinex理论表明低光照图像与正常图像存在点除关系,其中正常图像可通过低光照图像点除根据低光照图像生成的光照图像获得。

Jobson等人 \cite{DBLP:journals/tip/JobsonRW97a} 基于Retinex理论进行了一些基本的尝试,通过引入滤波器进行光照的估计,但获得了不符合真实自然图像分布的结果,出现了未知的伪影以及色偏等现象。考虑到在一些复杂场景下,噪声和伪影始终伴随着增强过程,因此Li等人 \cite{DBLP:journals/tip/LiLYSG18} 构建了一种基于Retinex的联合低光照增强和去噪的模型,并通过定义不同的先验约束来建立优化目标。然而,由于先验约束不强以及复杂的求解过程,这项工作时常会产生过度平滑和亮度不足的结果。

随着研究的深入,研究者们发现采用Retinex模型来实现亮度提升的关键在于光照层估计。Guo等人 \cite{DBLP:journals/tip/GuoLL17} 构建了第1个只考虑对光照进行建模并求解的工作,所提出方法命名为LIME,通过使用保留边缘的平滑方法RTV(relative total variation) \cite{DBLP:journals/tog/XuYXJ12} 优化了从输入得到的初始光照。不可否认的是,该项工作取得了显著的性能,亮度突出且结构明显。但是在大多数情况下会出现曝光过度的现象。为解决LIME存在的过曝现象,Zhang等人 \cite{DBLP:conf/mm/ZhangYXZZ18, DBLP:journals/tmm/ZhangNZXZ21}从不同角度引入了一系列的光照约束,成功地将过曝现象解决,但由于引入更多的正则项约束,导致算法求解过程复杂,其推理速度显著变慢。

总体而言,对于以上基于模型优化的方法,如何设计先验正则项是大部分工作的核心,其设计过程往往依赖于一系列对于现实环境的假设条件,先验表征能力有限。此外,已有工作始终需要针对实际情况进行手动调整诸多模型参数。因此这些基于模型的工作无法在某些具有挑战的场景中实现一致优异的性能。更重要的是,基于模型优化的方法往往使用迭代过程,因此相对耗时,不利于实际应用。


\noindent\textbf{基于深度学习的方法}~从实现目的来看,基于深度学习的低光照图像增强方法能够粗略地分为两类,用于亮度增强的方法以及联合亮度增强与噪声去除的方法。以下将从这两方面展进行介绍。

1) 用于亮度增强的方法。低光照图像增强的一个核心任务在于提升图像亮度以显示更多结构与细节,因此一系列专注于亮度增强的工作相继提出。早期的工作由于成对数据的匮乏,普遍采用合成数据的方式来进行深度网络的训练。Shen等人 \cite{DBLP:journals/corr/abs-1711-02488}将卷积神经网络与Retinex理论相结合,将多尺度Retinex看做是具有跳跃链接或者是残差形式的级联高斯卷积,设计出一个多尺度的卷积神经网络MSR-net(multi-scale Retinex network),并基于Photoshop处理后的成对数据获得端对端低光照图像增强网络。网络中采用对数变换的方式将Retinex模型由相乘的形式转换为相加。由于对数变换会抑制亮区域梯度的变化,该方法容易引起细节丢失。Li等人 \cite{DBLP:journals/prl/LiGPP18}提出了一种用于低光照图像增强的卷积神经网络(LightenNet)。他们通过基于Retinex理论创建的训练对来训练设计的网络结构,但是该方法仍然难以获得令人满意的增强效果,尤其是在一些有挑战性的真实场景。

2) 联合亮度增强与噪声去除的方法。以上提及的低光照图像增强算法着重于对于亮度的提升,忽略了对于一些恶劣场景下捕获图像存在的噪声问题。为提供更高的视觉质量,现有的主流工作将同时实现亮度增强与噪声去除作为低光照图像增强的核心。Lore等人 \cite{DBLP:journals/pr/LoreAS17}设计了一种低光照网络(low-light net,LLNet)深度自动编码器,在提高低光照图像对比度的同时兼顾去噪。但由于合成数据的不真实性,导致这种方法增强后的图像会不符合实际,且存在曝光不足的问题。除了关注于网络结构的启发式设计以外,一部分工作考虑引入额外信息来辅助实现低光照图像增强。Fan等人 \cite{DBLP:conf/mm/FanWY020}将额外的先验信息引入低光照图像增强任务中。该方法将语义分割网络与Retinex方法相结合,将语义分割的结果直接作用于反射层估计网络,间接作用于低光照图像增强。除了使用语义标签之外,Zhu等人 \cite{DBLP:conf/aaai/ZhuPCY20}将图像的边缘信息引入到低光照图像增强网络中,提出了一种边缘增强多重曝光融合网络。

此外也存在一些利用其他新颖技术来实现联合亮度增强与噪声去除的工作。Zhu等人 \cite{DBLP:conf/aaai/ZhuPCY20}提出了一种新颖的三分支卷积神经网络。与之前基于Retinex的方法略有不同,该网络将图像分解为3个分量,分别是光照、反射和噪声。在网络训练过程中,通过使用迭代最小化损失函数进行权重更新。Lim和Kim \cite{DBLP:journals/tmm/Lim021}利用拉普拉斯金字塔在图像空间和特征空间中的有用性,提出了一种称为深度堆叠拉普拉斯恢复(deep stacked Laplacian restorer,DSLR)的方法。该方法能够同时生成用于亮度提升的光照信息和用于结构增强的细节信息,并在图像空间中逐步整合光照和细节信息。由于特征空间中定义的拉普拉斯金字塔基于多尺度结构中高阶残差的丰富连接,使得恢复过程更加高效。Xu等人 \cite{DBLP:conf/cvpr/0010YYL20}构建了具有真实噪声的低光照图像数据集以及对应的清晰图像。基于此数据集,提出了一种基于频率的分解和增强模型,以同时实现噪声抑制和细节增强。

考虑到现有成对数据训练机制产生的泛化性能不足,以及现有成对数据自身存在的不精确性,一系列减轻对于成对数据依赖的工作相继提出。生成对抗网络(generative adversarial network,GAN) \cite{DBLP:conf/nips/LiuBK17}是一种代表性的非成对数据训练网络,且已在一系列图像到图像(如黑夜到白天)的转换中取得显著成功。因此一个直接的想法是,GAN可以用于解决低光照图像增强问题。Shi等人 \cite{DBLP:journals/corr/abs-1906-06027}提出了一个生成器,并利用转换的SID数据集 \cite{DBLP:conf/cvpr/ChenCXK18}来实现成对数据训练。但是,由于训练过程中过于关注分布,经常导致增强的结果看起来是不自然的。为了使得恢复出的增强图像更加自然,Guo等人 \cite{DBLP:conf/cvpr/GuoLGLHKC20}通过逐步推导构造出了一种像素级别的曲线估计卷积神经网络Zero-DCE(zero-reference deep curve estimation),并设计了一系列的零参考训练损失函数,以解决低光照图像增强问题。进一步地,Li等人 \cite{DBLP:journals/pami/LiGL22}提供了加速的版本Zero-DCE++,显著提升运算效率,性能几乎保持不变。最近,在注意力机制的启发下,Jiang等人 \cite{DBLP:journals/tip/JiangGLCFSYZW21}建立了一个具有自我注意力机制的生成对抗网络,并且以一种不成对的GAN的方式进行训练。尽管该方法的性能要远优于现有的一系列基于GAN的低光增强方法,但是由于忽略了物理原理的作用,该方法在增强过程中总是会产生一些未知的伪像。Zhang等人 \cite{DBLP:journals/corr/abs-2002-11300}通过设计多种与基于模型优化的方法相关的训练损失函数,并基于RetinexNet的体系结构进行重新设计与调整,进而建立了一个自监督学习的卷积神经网络,以同时输出光照和反射。

现有低光照图像增强技术在方法层面上已经完成从传统模型设计到数据驱动的深度学习的跨越。在学习机制方面,正在从全监督学习迈向半监督/无监督学习;在应用场景方面,从相对简单的场景逐步转向更加具有挑战的真实场景(如手机拍摄);从评估方式来看,现有技术正在渐渐地跳出基于视觉质量的感知评估体系,而愈发关心下游高层视觉任务的性能,逐步从只关注视觉质量转变为高层视觉任务性能优先。此外,随着视频在生活中越来越常见,现有的工作也在逐步从空间上的图像层面转变为时序上的图像层面,即视频处理。
